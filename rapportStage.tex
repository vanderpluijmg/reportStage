\documentclass{article}

\title{Rapport de stage}
\author{Gregory van der Pluijm}
\date{\today}

\begin{document}
\pagenumbering{gobble}
\maketitle
\clearpage
Volume 1
\clearpage
\pagenumbering{arabic}
Remerciement
\clearpage
\tableofcontents
\clearpage
\section{Abreger du contenu du rapport}
\section{Introduction}
\section{Environemment de stage}
L’entreprise que j’ai intégrée possède un open space, situer à Watford dans un business parc, qui est accessible via un bus qui passe par la gare centrale. Au niveau du matériel informatique, j’ai eu le droit à un laptot avec Windows dessus. La majorité des applications de PSI tournent sur ce systeme d’opération. Dès que j’ai commencé à faire du vrai développement, j’ai demandé une partition Linux, ils m’ont donc donné un desktop avec Ubuntu. Au début du stage, je venais tous les jours au bureau, à part le vendredi ou personne n’est là. C’est dans cette période que j’ai appris comment l’entreprise fonctionnait ; comment la communication se fait au sein de l’entreprise. Ensuite, j’ai commencé à m’adapter aux horaires des gens que j’avais besoin. Je m’assurais d’être au bureau en même temps qu'eux pour pouvoir poser des question facilement si necessaire. C’était donc plus realistiquement 2 à 3 jours au bureau par semaine et le reste en télé-travaille.
\newline
\newline
Sachant qu’il y a deux sections differentes au sein de PSI Metals UK j’ai eu la chance de pouvoir travailler avec chaque. J’ai donc eu deux projets different, deux environnements informatique different a m’adapter simultanément. 

En premier, j’ai dû me familiariser avec Zoho Creator qui est un logiciel basé sur le cloud pour créer des applications personnalisées gourmande en données. Le deuxième environnement était le développement d’une application en Java.

D’abord, Zoho était pour moi l’environnement que j’appréhendais le plus. Inconnu de base, j’ai dû apprendre plusieurs aspects en même temps. Pour ce, j’ai eu droit à une application déjà créer par l’entreprise mais en sandbox, voulant dire que je pouvais toucher a plein de configuration differente sans que les changements prennent vraiment lieux. Ceci m’a beaucoup aider pour comprendre le fonctionnement du logiciel, mais aussi de Deluge, un language propriétaire de script. De plus, vu que la migration de toute leurs donner vers Zoho a été fait avant que je commence mon stage, j’ai eu besoin d’un certain temps pour me retrouver dans ce qui existait déjà. Pour cela je me suis de nouveau dirigé vers le sandbox pour voir comment certaines de leurs applications interagissaient ensemble afin de pouvoir interagir avec et comprendre les nombreuses bases de données qu’il y avait. J’ai d’abord dû comprendre ce que la section Global Support faisait concrètement. Sur Zoho, ils essayent d’automatiser un tas de choses, par exemple l’envoi d’email pour notifier la date d’expiration d’une licence ou encore automatiquement tirer les informations nécessaire d’une license,  de représenter les donners de plusieurs manières afin d’avoir une vision plus claire notamment grace à la réalisation de different dashboards. En plus, ils sont en plein developpement du’une app store pour les clients qui utilisent de l’information present sur Zoho. J’ai eu, par exemple l’opportuniter de créer une maquette graphique du dashboard d’un client. J’ai pour cela utiliser Figma.  

L’environnement du deuxieme projet etais du Java. Sachant que j’avais deja fait des projets au paravant avec ce langague, j’avais plus de faciliter. Une partie des developpeur de la boite se charge de develloper un IDE qui s’apelle Auto-test. Cet IDE permet entre autre aux utilisateur de scripter des tests d’application d’interface graphique qui utilise le framework .Net. Auto-test utilise pour cela son propre language XML avec lequel ont peut par exemple se connecter a une base de donner, faire des requete SQL et s’assurer des bonnes valeurs. Ils m’ont tacher des reproduire les meme fonctionnement pour les application .Net avec les applications qui utilisent les framework Swing et JavaFx. J’ai pour cela d’abord du rechecher beaucoup sur le fonctionnement du language et de la Java Virtual Machine. J’ai pu avoir l’aide du lead developpeur qui travaille sur Auto-test.
\section{Presentation du travaille}
\section{Conclusion}
\section{Index et glossaire des terme utiliser}

\end{document}